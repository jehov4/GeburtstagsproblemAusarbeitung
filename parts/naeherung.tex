%! Author = Arbeit
%! Date = 26.03.2021

% Preamble
\documentclass[11pt]{article}

% Packages
\usepackage{amsmath}

% Document
\begin{document}
    \section{Näherung}

    Das offenbar spektakuläre Ereignis besteht scheinbar darin, dass bei der schrittweisen rein zufälligen Belegung
    von \(n\) = 365 unterschiedlichen Tagen die erste Kollision bereits nach dem 23. Tag stattfand, d.h. ein bereits
    durch einen Geburtstag belegter Tag wurde erneut belegt. Intuitiv würde man erwarten, dass der Zeitpunkt dieser
    ersten Kollision viel später liegt. Würden Sie z. B. bei \(n\)= 1000 Tagen darauf wetten, dass die erste Kollision
    spätestens nach 50 Personen auftrat?
    \newline
    Zur Modellierung des Kollisionsphänomens betrachten wir die Zufallsvariable
    \newline
    \[{ X_{n} = \text{ Zeitpunkt der ersten Kollision beim sukzessiven \newline rein zufälligen Besetzen von n Tagen. } }\]
    \newline
    Da mindestens \(2\) und höchstens \(n + 1\) Versuche bis zur ersten Kollision nötig sind, nimmt \(X_n\) die Werte \(2,3, . . ., n + 1\) an, und es gilt


    \[\mathbb{P}(X_{n} \geq k+1) = \frac {n*(n-1)*(n-2)* ... * (n-k + 1)}{n^k} = \frac{n^{\underline{\ k}}}{n^k}\]
    \newline\newline
    für jedes \(k = 1,2,...,n+1\).
    \newline
    Daraus folgt das Gegenereignis:
    \newline\newline
    \[\mathbb{P}(X_{n} \leq k) = 1-\prod \limits_{j=1}^{k-1}(1- \frac{j}{n})\]

    \[(k=2,3,...,n+1; \mathbb{P}(X_n\leq1) = 0)\]
    \newline
    Die Formel zeigt die Wahrscheinlichkeiten \(\mathbb{P}(X_{n} \leq k)\) als Funktion von \(k\) für den Fall \(n = 365\).
    Für das Ereignis \({X_{n} \leq 23}\) gilt\( \mathbb{P}(X_{n} \leq 23) = 0,5073 \)
    \newline
    \newline
    Es vermag vermutlich überraschend erscheinen, dass wir bei fast 365 möglichen Tagen im Jahr durchaus auf das
    Auftreten des doppelten Geburtstags nach höchstens 23 Personen wetten können. Doch der Grund hierfür ist, dass wir
    nach irgendeiner beliebigen und nicht nach einer bestimmten Kollision suchen.
    Der Zeitpunkt der ersten Kollision bei der rein zufälligen sukzessiven Besetzung von \(n\) Tagen ist von
    der Größenordnung \(\sqrt{n}\).
    \newline
    \newline
    Da beim Geburtstagsproblem  \(\mathbb{P} (X365 \leq 23) = 0.507 > 1/2\) gilt, kann durchaus darauf gewettet werden,
    dass unter 23 aber auch mehr Personen mindestens zwei am gleichen Tag Geburtstag haben. Bei
    \(\sqrt{365 \cdot 2 \cdot ln(2)} \approx 22.49\)
    ist die Approximation schon für \(n\) = 365 sehr gut.


\end{document}
