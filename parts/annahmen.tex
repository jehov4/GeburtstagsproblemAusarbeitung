%! Author = Arbeit
%! Date = 26.03.2021

% Preamble
\documentclass[11pt]{article}

% Packages
\usepackage{amsmath}

% Document
\begin{document}
    \section{Annahmen}

    Um das Geburtstagsparadoxon als vereinfachtes statistisches Beispiel zu verwenden, werden folgende drei Annahmen getroffen:

    \begin{itemize}
        \item Jedes Jahr hat einheitlich 365 Tage, ohne Berücksichtigung des Schaltjahres.
        \begin{itemize}
            \item Das Schaltjahr findet nur alle vier Jahre Anwendung und wäre im Experiment nicht so einfach zu berücksichtigen. Diese Annahme hat natürlich aber auch eine geringfügig größere Wahrscheinlichkeit zur Folge.
        \end{itemize}
    \end{itemize}

    \begin{itemize}
        \item Jeder der 365 Tage eines Jahres ist als Geburtstag gleich wahrscheinlich.

        \begin{itemize}
            \item Das Jahr erfüllt in unserer Variante die Anforderungen an ein "Laplace"-Experiment.
        \end{itemize}

        \begin{itemize}
            \item In der Realität ist das nicht so, denn hier gibt es eine Häufung an Geburten nach besonderen Ereignissen. Ein Beispiel ist der Valentinstag.
        \end{itemize}

    \end{itemize}

    \begin{itemize}
        \item Das Auswahlkriterium der Testpersonen erfolgt hinsichtlich ihres Geburtstages.
    \end{itemize}

    \begin{itemize}
        \item Das Ereignis "mindestens ein doppelter Geburtstag” ist schwierig zu berechnen, da es eine große Menge an Teilereignissen einschließt (u.a. drei Personen haben den gleichen Geburtstag oder zwei Paare haben denselben Geburtstag). Daher wird in dieser Ausarbeitung das Gegenereignis “kein doppelter Geburtstag” angenommen.
    \end{itemize}

    \begin{itemize}
        \item Als kleinste natürliche Zahl wird \(1\) angenommen.
    \end{itemize}

Auf diesen Annahmen beruhen alle Berechnungen in dieser Ausarbeitung.


\end{document}
