% !TeX root = ../main.tex
\documentclass[../main.tex]{subfiles}
\begin{document}

\section{Herleitungen der verwendeten Formeln}
\begin{flushleft}
Zur Modellierung des Geburtstagsproblems betrachten wir die Zufallsvariable:
    \begin{equation}
      X_{ n } := \text{Zeitpunkt der ersten Kollision bei n Personen mit rein zufällig gewählten Geburtstagen} \label {first}
    \end{equation}




Da zumindest zwei Personen vorhanden sein müssen damit es zu einer Kollision kommt ist der minimale Wert $2$. Höchstens sind es $n + 1$ Personen. Somit nimmt $X_{ n }$ die Werte $2,3,... ,(n+1)$ an und es gilt:
   \begin{equation}  \label{eq:01}
   \mathbb{P}(X_{ n } \geq k + 1)  = \frac{ n * (n - 1) * (n - 2) * ...* (n - k + 1) }{ n^k }
    \end{equation}

für jedes $k = 1,2,...,n + 1$. Durch die Annahme der gleichen Verteilung der Zufallsereignisse (Laplace-Modell), ergibt der Zähler von (\autoref{eq:01}) die Anzahl der günstigen Fälle an. \newline

Aus \autoref{eq:01} folgt durch Verwendung des Gegenereignisses:
\begin{eqnarray}
\mathbb{P}(X_{ n } \geq k) = 1 - \prod_{ j=1 }^{k-1}{ (1 - \frac{ j }{ n }) }  \label{eq:02.1} \\
\mathbb{P}(Xn\leq 1) = 0  \label{eq:02.2}
\end{eqnarray}

Da bei einer einzigen Person $k = 1$ keine Kollision auftreten kann ist die Wahrscheinlichkeit für dieses Ereignis 0 (\autoref{eq:02.2}). Deshalb ist der Wertebereich für $k$ mit $k = [2;n+1]$ angegeben. \newline

In der Abbildung xxx ist die Wahrscheinlichkeit $P(X_{n} \leq k)$ durch eine Funktion von k mit dem Parameter $n = 365$ dargestellt. Unterschiedliche Zahlenwerte für $n$ sind in Tabelle xxx aufgeführt. Für das Ereignis $ X_{n} \leq 23 $ ist die Wahrscheinlichkeit bereits höher als 50\%. \newline

Auf den ersten Blick scheint überraschen das bei $8,5 * 10^{58}$ möglichen Kombinationen ($365^{23}$) der Geburtstagsverteilung bei 23 Personen. Die Wahrscheinlichkeit eines doppelten Geburtstags schon über 50\% liegt. \newline

Die Erklärung hierfür ist das wir auf irgendeine und nicht auf eine bestimmte Kollision warten. Es soll im folgenden gezeigt werden das die erste Kollision bei zufälliger Besetzung von $n$ Tagen von der Größenordnung $\sqrt{n}$ ist. \newline

((((((Warum ist die Größenordnung sqrt(n) hier einfügen)))))) \newline

Henze Seite 70 Satz 10.1  \newline


Warum verwendung der ungleichung Mittelwertsatz beweis
\begin{equation}
1 - x <= e^{-x}
\end{equation}

\subsection{Vereinfachung der Wahrscheinlichkeitsverteilung}

Unter Verwendung der Ungleichung
\begin{equation}
1 - x \leq e^{ -x } (x \in \mathbb{R})
\end{equation}




ist es uns möglich die unter \autoref{eq:02.1} angegebene Funktion so weit zu vereinfachen, das kein Produkt- oder Summenzeichen mehr vorhanden ist, dadurch sind weitere Berechnungen einfacher zu realisieren.
\begin{eqnarray}
\mathbb{P}(X_{ n } \leq k) \approx 1 - \prod_{ j = 1 }^{ k - 1 }{ 1 - \frac{ j }{ n } } \geq 1 - exp( - \sum_{ j = 1 }^{ k - 1 }{ \frac{ j }{ n } } )  \label{eq:03.1}\\
\approx 1 - exp( - \frac{ k (k + 1) }{ 2n } )  \label{eq:03.2}
\end{eqnarray}

Bei der ersten Umformung in \autoref{eq:03.1} wird das Produkt zu einer Summe im Exponenten von $e$, aufgrund der allgemeinen Potenzgesetze  $a^{ r } * a^{ s } = a^{ r + s }$. In der nächsten Umformung wird die im ersten Schritt geschaffene Summe mithilfe der Gausschen Summenformel ersetzt. \newline

Somit ergibt sich zur Berechnung der Wahrscheinlichkeit die allgemeine Formel:
\begin{equation}
\mathbb{P}(X \leq k) \approx 1 - exp( - \frac{ k (k + 1) }{ 2n } ) \label{eq:04}
\end{equation}
\subsection{Wahrscheinlichkeitsdichte}

In der Stochastik beschreibt die Wahrscheinlichkeitsdichte ist eine spezielle reellwertige Funktion zur Konstruktion von Wahrscheinlichkeitsverteilungen.

Im Unterschied zu Wahrscheinlichkeiten kann die Wahrscheinlichkeitsdichte auch größere Werte als 1 annehmen. Dabei wird nicht der Funktionswert sondern die Fläche unterm Funktionsgraphen berechnet, also das Integral. \newline

Die Dichte kann mit zwei verschiedenen Herangehensweisen konstruiert werden: Durch eine Funktion die aus der Wahrscheinlichkeitsverteilung generiert wird, oder durch die Ableitung der Wahrscheinlichkeitsverteilung. Es unterscheidet sich nur die Herangehensweise. \newline

Im Weiteren wird nur noch auf den Fall eingegangen, die in dem die Dichte aus der Wahrscheinlichkeitsverteilung abgeleitet wird. \newline

Allgemein ist die Wahrscheinlichkeitsdichte dann folgendermaßen definiert:
\begin{equation}
\mathbb{P}(]-\infty, a])= \int_{-\infty}^a f(x) \, \mathrm d x
\end{equation}
bzw.
\begin{equation}
\mathbb{P}(X \leq a)= \int_{-\infty}^a f(x) \,\mathrm d x
\end{equation}

Für das Geburtstagsproblem ist die Wahrscheinlichkeitsverteilung aus Formel \autoref{eq:04} als gegeben anzusehen. Es lässt sich aus der Definition für die Dichte folgende Gleichung ableiten:
\begin{eqnarray}
\mathbb{P}(X \leq k) \approx \int_{ -\infty }^{ k }{ f(k) dk } \\
\implies 1 - exp( - \frac{ k (k + 1) }{ 2n } ) \approx \int_{ -\infty }^{ k }{ f(k) dk }
\end{eqnarray}

Durch ableiten der Gleichung ergibt sich eine Definition für die Funktion f(k):

\begin{eqnarray}
\implies 1 - exp( - \frac{ k (k + 1) }{ 2n } ) \approx \int_{ -\infty }^{ k }{ f(k) dk } \quad | \quad \frac{ d }{ dk }\\
\implies \frac{ (k-1) }{ n } * exp( - \frac{ k (k + 1) }{ 2n } ) \approx f(k)
\end{eqnarray}

Somit ist die Dichte für das Geburtstagsproblem folgendermaßen definiert:

\begin{equation}
\mathbb{P}(X = k) \approx \frac{ (k-1) }{ n } * exp( - \frac{ k (k + 1) }{ 2n } )
\end{equation}

 \subsection{Erwartungswert}

Der Erwartungswert für das Geburtstagsproblem lässt sich mithilfe der Weibull-Verteilung konstruieren. In dieser Ausarbeitung wird nicht näher auf die Weibull-Verteilung oder die verwendete Gamma-Funktion eingegangen, diese werden als vorausgesetzt angesehen.  \newline

Für die um 1 nach rechts verschobene Weibull-Verteilung werden folgende Parameter verwendet. $k = 2$ dadurch ergibt sich eine Rayleigh-Verteilung und $\lambda = \frac{ 1 }{ \sqrt{2n} }$. $\mathcal{T}$ bezeichnet die Gammafunktion. \newline

Somit ergibt sich die Funktion:
\begin{equation}
\mathbb{E}(X - 1) = \mathbb{E}(X) - 1 \approx \sqrt{ 2n } * \mathcal{T} (1 + \frac{ 1 }{ 2 } ) = \sqrt{ 2n } * \frac{ \sqrt{ \pi } }{ 2 } = \sqrt{ \frac{ \pi*n }{ 2 } }
\end{equation}
\begin{equation}
\mathbb{E}(X) \approx \sqrt{ \frac{ \pi*n }{ 2 } } + 1
\end{equation}

für den Erwartungswert.

 \subsection{Varianz}

Die Varianz wird so wie der Erwartungswert auch mithilfe der Weibull-Verteilung bestimmt. Die Parameter sind hierbei die selben wie schon für den Erwartungswert verwendet wurden.
\begin{eqnarray}
Var(X) \approx \frac{ 1 }{ \lambda^{ 2 } } * [\mathcal{T}(1+\frac{ 2 }{ k }) - \mathcal{T}^{ 2 }(1+\frac{ 1 }{ k } )]   \label{eq:08}\\
\approx 2n * [\mathcal{T}(1+\frac{ 2 }{ 2 }) - \mathcal{T}^{ 2 }(1+\frac{ 1 }{ 2 } )] \\
\approx 2n * [1 - \frac{ \pi }{ 4 }]
\end{eqnarray}

Die Definition der Varianz der Weibull-Verteilung ist in \autoref{eq:08} gegeben.

 \subsection{Quantile}

Ein Quantil ist ein Lagemaß in der Statistik. Den meisten ist der Median bekannt, dabei handelt es sich um das 50\% oder $\frac{1}{2}$ Quantil. Es lassen sich aber auch beliebige Quantile zwischen 0 und 1 bestimmen. Allgemein sind Quantile Schwellenwerte. Werden die gegebenen Daten nach ihrer Wertigkeit sortiert, ist ein bestimmter Anteil kleiner als das Quantil. \newline

Gegeben sei eine beliebige Zufallsvariable $X$. Dann ist $x_{p}$ das $p$-Quantil von $X$, wenn gilt:

\begin{equation}
\mathbb{P}(X \leq x_p) \geq p
\end{equation}
und
\begin{equation}
\mathbb{P}( x_p \leq X ) \geq 1- p
\end{equation}

Im Folgenden wird beschrieben wie aus dieser Definition eine Funktion konstruiert werden kann mit der sich die Quantile für das Geburtstagsparadoxon bestimmen lassen. \newline

Für das  $\frac{1}{2}$ Quantil (Median):
\begin{align*}
& \mathbb{P}(X \leq k) = 1 - \mathbb{P}(X > k) \geq 1 - exp(-\frac{ k(k-1) }{ n }) \overset{!}{=} \frac{ 1 }{ 2 } \\
& \Leftrightarrow ln(\frac{ 1 }{ 2 }) = -\frac{ k(k-1)}{ 2n } \\
& \Leftrightarrow -2n*ln(2) = -k(k-1) \\
& \Leftrightarrow 2n*ln(2) = k(k-1) \\
& \Leftrightarrow 2n*ln(2) = k^{2} - k \\
& \Leftrightarrow - k^{2} + k + 2n*ln(2) = 0 \\
& \Leftrightarrow k^{2} - k - 2n*ln(2) = 0 \\
& \Leftrightarrow k = \frac{ 1 }{ 2 } \pm \sqrt{ \frac{ 1 }{ 4 } - 2n*ln(2)} \\
\end{align*}

Daraus lässt sich dann folgendes Ableiten:
\begin{equation}
Q_{ \frac{ 1 }{ 2 } }(X) \leq \bigg( \frac{ 1 }{ 2 } + \sqrt{ \frac{ 1 }{ 4 } - 2nln(2)} \bigg) \leq \bigg(1+\sqrt{ 2nln(2) }\bigg)
\end{equation}
\newline

Das Quantil befindet sich somit in den Grenzen der Quadratischen Funktion $ k = \frac{ 1 }{ 2 } \pm \sqrt{ \frac{ 1 }{ 4 } - 2nln(2)} $. \newline

Das spezielle $ \frac{1}{2} $-Quantil lässt sich auch allgmein bestimmen, sodass der Schwellwert $p$ ein Parameter der Funktion ist:

\begin{align*}
& 1 - exp(-\frac{ k(k-1) }{ n }) \overset{!}{=} p \\
& \Leftrightarrow ln(p) = -\frac{ k(k-1)}{ 2n } \\
& \Leftrightarrow 2n*ln(p) = -k(k-1) \\
& \Leftrightarrow 2n*ln(p) = k(k-1) \\
& \Leftrightarrow 2n*ln(p) = k^{ 2 } - k \\
& \Leftrightarrow - k^{ 2 } + k + 2n*ln(p) = 0 \\
& \Leftrightarrow k^{ 2 } - k - 2n*ln(p) = 0 \\
& \Leftrightarrow k = \frac{ 1 }{ 2 } \pm \sqrt{ \frac{ 1 }{ 4 } + 2n*ln(p)}
\end{align*}

Daraus lässt sich wie bei der speziellen Lösung, folgende Aussage ableiten:
\begin{equation}
Q_{ p }(X) \leq \bigg( \frac{ 1 }{ 2 } + \sqrt{ \frac{ 1 }{ 4 } - 2nln(p)} \bigg) \leq \bigg(1+\sqrt{ 2nln(p) }\bigg)
\end{equation}

Somit lassen sich die Quantile durch folgende Funktionen approximieren:

\begin{equation}
Q_{ p }(X) \approx \frac{ 1 }{ 2 } + \sqrt{ \frac{ 1 }{ 4 } - 2nln(p)}
\end{equation}

\begin{equation}
Q_{ p }(X) \approx 1+\sqrt{ 2nln(p) }
\end{equation}







\end{flushleft}

\end{document}
