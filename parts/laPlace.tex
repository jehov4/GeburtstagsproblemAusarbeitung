%! Author = Arbeit
%! Date = 26.03.2021

% Preamble
\documentclass[11pt]{article}

% Packages
\usepackage{amsmath}

% Document
\begin{document}
    \section{Gleichverteilung:}

    Sei \Omega eine endliche Ergebnismenge und \(p : \Omega \rightarrow [0, 1]\) eine Wahrscheinlichkeitsfunktion.
    Wenn \( \mathbb{P}(x) = \frac{1}{|\Omega|}\) gilt, dann spricht man von Gleichverteilung,
    genauer: die Wahrscheinlichkeitsfunktion \(p\) ist gleichverteilt.
    Alle möglichen Ausgänge des Zufallsexperiments treten mit der gleichen Wahrscheinlichkeit auf.


    \section{Laplace:}

    Ein Versuch heißt dann "Laplace-Versuch", wenn keines der Elementarereignisse mit einer größeren
    Häufigkeit auftritt als ein anderes. Somit ist die Wahrscheinlichkeit aller möglichen Ergebnisse gleich.

    \[\mathbb{P}(x)= \frac{1}{\Omega}\]

    Bei einem "Laplace"-Versuch wird die Wahrscheinlichkeit \((E)\) eines Ereignisses E wie folgt berechnet:
    \[\mathbb{P}(E)= \frac{\text{Anzahl der zum Ereignis E gehörenden Ergebnisse}} {\text{Anzahl aller möglichen Ergebnisse}} = \frac{|E|} {|\Omega|}\]


    Beispiel:
    \begin{itemize}
        \item Werfen einer fairen Münze (\(\Omega= {Kopf, Zahl}\))
    \end{itemize}
    \begin{itemize}
        \item n-maliges Werfen eines fairen Würfels (\(\Omega=({1,\cdots,6}) ^n\))
    \end{itemize}


    \section{Gaußsche Summenformel: }

    Mithilfe der Gaußschen Summenformel lässt sich die Summe beliebig vieler (= \(n\)) natürlicher Zahlen berechnen.
    Hierbei addiert man alle natürlichen Zahlen von \(1\) bis zu der gewählten Grenze \(n\).

    \[1 +2 +3+ 4+ \cdots +n = \sum \limits_{k=1}^{n} k = \frac{n(n+1)}{2} = \frac {n^{2}+n}{2}\]

\end{document}
