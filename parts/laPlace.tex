%! Author = Arbeit
%! Date = 26.03.2021

% Preamble
\documentclass[11pt]{article}

% Packages
\usepackage{amsmath}

% Document
\begin{document}
    \section{Gleichverteilung:}

    Sei Ω eine endliche Ergebnismenge und \(p : \Omega \rightarrow [0, 1]\) eine Wahrscheinlichkeitsfunktion.
    Wenn \( p(x) = \frac{1}{|\Omega|}\) gilt, dann spricht man von Gleichverteilung,
    genauer: die Wahrscheinlichkeitsfunktion \(p\) ist gleichverteilt.
    Alle möglichen Ausgänge des Zufallsexperiments treten mit der gleichen Wahrscheinlichkeit auf.


    \section{La Place:}

    Ein Versuch heißt dann ”La Place Versuch”, wenn keines der Elementarereignisse mit einer größeren
    Häufigkeit auftritt als ein anderes. Somit ist die Wahrscheinlichkeit aller möglichen Ergebnisse gleich.

    \[P(x)= \frac{1}{\Omega}\]

    Bei Laplace Versuchen wird die Wahrscheinlichkeit \(P(E)\) eines Ereignisses E wie folgt berechnet:
    \[P(E)= \frac{Anzahl der zum Ereignis E gehörenden Ergebnisse} {Anzahl aller möglichen Ergebnisse} = \frac{IEI} {ISI}\]


    Beispiel:
    \begin{itemize}
        \item Werfen einer fairen Münze (\(\Omega= {Kopf, Zahl}\))
    \end{itemize}
    \begin{itemize}
        \item N-maliges Werfen eines fairen Würfels (\(\Omega=({1,...,6}) ^n\))
    \end{itemize}


    \section{Gaußsche Summenformel: }

    Mithilfe der Gaußschen Summenformel lässt sich die Summe beliebig vieler (= \(n\)) natürlicher Zahlen berechnen.
    Hierbei addiert man alle natürlichen Zahlen von 1 bis zu der gewählten Grenze \(n\).

    \[0 + 1 +2 +3+ 4+ ...+n = \sum \limits_{k=0}^{n} k = \frac{n(n+1)}{2} = \frac {n^{2}+n}{2}\]

\end{document}
