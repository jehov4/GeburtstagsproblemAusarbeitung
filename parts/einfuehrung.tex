 % !TeX root = ../main.tex
\documentclass[../main.tex]{subfiles}
\begin{document}
\section{Einführung}

Das Geburtstagsproblem generiert eine Warschleinlichket das unter k zufällig gewählten Personen mindestens zwei am selben Tag geburtstag haben. Es ist eine Abwandlung des Paradoxons der ersten Kollision. \cite{henze} Bei k=23 ist die warscheinlickteit das zwei personen bereits am gleichen tag geburtsagt haben bei über 50\%. Das ist erstaunlich stochastisch weniger bewanderte personen
fürden stehts subjektiv zu einer sehr viel geringeren warscheinlichkeit kommen. In der folgenden ausarbeotung wrrden wir auf die näherung für das paradoxon eingehen sowie quanitle und den erwartungswert.

das an irgendeinem Tag im Jahr irgendeine der k personen

zuerst bestimmen wir n. Das ist die warscheinlichkeit das eine person an einem bestimmten Tag der jahren geburtstag hat. Das gewählte Jahr besitzt 365 Tage (Auf Schaltjahre wird nicht nachgegangen). Außerdem gehen wir von gleich gewichteten Tagen aus.

Daraus ergibt sich:

\begin{equation}
n = \frac{1}{365}
\end{equation}

Um auszurechnen, wie viele Personen sich in einem Raum befinden müssen, so dass die Wahrscheinlichkeit dass mindestens zwei Personen am selben Tag Geburtstag haben bei 50\% oder meht liegt verwenden wir das Gegenereignis. Wir berechnen die Wahrscheinlichkeit, dass alle Personen im Raum an verschiedenen Tagen Geburtstag haben und nähern uns den von oben den 50\%

Für 2 Personen:

\begin{equation}
 \frac{365}{365} * \frac{364}{365} = 0,997
\end{equation}

Die erste Person kann aus 365 Tagen wählen ohne dass es zu einer Kollision kommt, für die 2. Person bleiben 364 Tage

Für 3 Personen:

\begin{equation}
 \frac{365}{365} * \frac{364}{365} * \frac{363}{365} = 0,991
\end{equation}

Dies wird weitergeführt, bis die Wahrscheinlichkeit für das Gegenereignis bei etwa 50\% liegt, somit liegt dann auch das Ereignis, dass mindestens Zwei Personen am selben Tag Geburtstag haben bei etwa 50\%. Dieser Fall Tritt bei einer Personen Zahl von 23 ein.

\begin{equation}
 \frac{365}{365} * \frac{364}{365} * \frac{363}{365} * ... * \frac{343}{365} = 0,493
\end{equation}

Die Wahrscheinlichkeit für keine Kollision liegt bei 0,493, somit liegt die Wahrscheinlichkeit für eine Kollision bei 0,507

\begin{equation}
 1-0,492 = 0,507
\end{equation}



\end{document}

