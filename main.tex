\documentclass{article}
\usepackage{etc/packages}
\numberwithin{equation}{section}


\addbibresource{main.bib}
\graphicspath{graphics/}

\makeglossaries
\input{etc/glossary}

\begin{document}
\subfile{etc/titlepage}

\pagenumbering{roman}

\begin{abstract}
Angenommen wir betrachten das Sommermärchen 2006. Hierbei durfte jede Mannschaft 23 Spieler für ihren Kader nominieren.
Der Kader der deutschen Fußballnationalmannschaft für dieses WM-Turnier bestand unter anderem aus Mike Hanke und
Christoph Metzelder. Obwohl die Beiden weder für den gleichen Verein spielten noch die gleiche Position im Feld
innehatten, haben sie dennoch etwas gemeinsam: ihren Geburtstag! Ein Zufall ist dies schon, aber ist es auch ein
seltenes Ereignis? Wie oft kommt so etwas vor?
Um dieser Frage auf den Grund zu gehen, eignet sich ein Blick auf das sogenannte Geburtstagsproblem, das auch unter
dem Begriff ”Geburtstagsparadoxon” zu finden ist. Seinen Namen erhielt es aufgrund der Tatsache, dass viele Menschen
es für eine scheinbar unsinnige Behauptung hielten, da sie nur ihr näheres Umfeld betrachteten und eine mögliche
Ansammlung von 23 beliebigen, sich zufällig an einem Platz befindenden Menschen, völlig außer Acht ließen.
Das Geburtstagsparadoxon gibt eine Antwort auf die Frage, wie viele Personen in einem Raum sein müssen, damit eine
bestimmte Wahrscheinlichkeit besteht, dass mindestens zwei Personen den gleichen Geburtstag haben. Somit generiert
es eine Wahrscheinlichkeit, dass unter k zufällig gewählten Personen mindestens zwei am selben Tag Geburtstag haben.
In der folgenden Ausarbeitung wird das Geburtstagsparadoxon zunächst hergeleitet. Daraufhin folgt eine Darstellung
der Näherung. Im Anschluss wird die Quantile beleuchtet und der Erwartungswert angegeben.
\end{abstract}

\clearpage

{\tableofcontents}
\setcounter{tocdepth}{3}
\printglossary
\printglossary[type=\acronymtype]

\clearpage


\pagenumbering{arabic}

\subfile{parts/einfuehrung}

\subfile{parts/annahmen}

\subfile{parts/laPlace.tex}

\subfile{parts/herleitungen}

\subfile{parts/naeherung.tex}

\subfile{parts/konkreteZahlen}




%\section{Einf\"uhrung}
%\paragraph{Herangehensweise}
%\ref{conclusions}
%\section{Dokumentenklassen} \label{documentclasses}

%\begin{itemize}
%\item article
%\item book 
%\item report 
%\item letter 
%\end{itemize}

%\section{Fazit}\label{conclusions}
%Nach langer Suche hat sich herausgestellt, dass es kein l\"angeres 
%\LaTeX{} Beispiel, als das von \cite{doe} geschriebene gibt. 
\clearpage
\printbibliography
\end{document}
